% use this for typesetting a chapter without a number, e.g. intro and outro
\def\chapwithtoc#1{\chapter*{#1}\addcontentsline{toc}{chapter}{#1}}

% If there is a line/figure overflowing into page margin, this will make the
% problem evident by drawing a thick black line at the overflowing spot. You
% should not disable this.
\overfullrule=3mm

% The maximum stretching of a space. Increasing this makes the text a bit more
% sloppy, but may prevent the overflows by moving words to next line.
\emergencystretch=1em

\theoremstyle{plain}
\newtheorem{theorem}{Theorem}
\newtheorem{lemma}[theorem]{Lemma}
\newtheorem{claim}[theorem]{Claim}
\newtheorem{definition}{Definition}
\theoremstyle{remark}
\newtheorem*{corrolary}{Corollary}

% real/natural numbers
\newcommand{\R}{\mathbb{R}}
\newcommand{\N}{\mathbb{N}}

% asymptotic complexity
\newcommand{\asy}[1]{\mathcal{O}(#1)}

% listings and default lstlisting config (remove if unused)
\DeclareNewFloatType{listing}{}
\floatsetup[listing]{style=ruled}

\DeclareCaptionStyle{thesis}{style=base,font={small,sf},labelfont=bf,labelsep=quad}
\captionsetup{style=thesis}
\captionsetup[algorithm]{style=thesis,singlelinecheck=off}
\captionsetup[listing]{style=thesis,singlelinecheck=off}

% Customization of algorithmic environment (comment style)
\renewcommand{\algorithmiccomment}[1]{\textcolor{black!25}{\dotfill\sffamily\itshape#1}}

% Here's how you define a custom language, Effekt is used here as an example
\lstdefinelanguage{Effekt}{
  morekeywords={%
    let,true,false,val,var,if,else,while,type,effect,interface,%
    try,with,case,do,fun,match,def,module,import,export,extern,%
    include,record,box,unbox,return,region,resource,%
    new,and,is,namespace,pure},
  otherkeywords={=>},
  alsoletter={?,!},
  sensitive=true,
  morecomment=[l]{//},
  morecomment=[n]{/*}{*/},
  morestring=[b]",
  morestring=[b]',
  morestring=[b]"""
}[keywords,comments,strings]

\floatname{listing}{Listing}
\lstset{ % use this to define styling for any other language
  language=Effekt,
  tabsize=2,
  showstringspaces=false,
  basicstyle=\footnotesize\tt\color{black!75},
  identifierstyle=\bfseries\color{black},
  commentstyle=\color{green!50!black},
  stringstyle=\color{red!50!black},
  keywordstyle=\color{blue!75!black}}

\newcolumntype{P}[1]{>{\centering\arraybackslash}p{#1}}

\lstMakeShortInline[columns=fixed]|
