\chapter{Existing ECS and Game Engines}\label{chap:engines}

\section{ECS concept}

A data-driven entity-based architecture with component databases for games was first implemented by the game `Thief: The Dark Project' (Looking Glass Studios, 1998). It was also used in some of their sequels and later games. After that, Scott Bilas from `Gas Powered Games' presented the approach they used for their `Dungueon Siege' game, with the title `A Data-Driven Game Object System' in 2002 (https://www.gamedevs.org/uploads/data-driven-game-object-system.pdf). This started to increase the attention around the ECS/data-driven architecture for games. In 2007 Mick West from Neversoft also talked about their ECS in the article `Evolve Your Hierarchy' (https://cowboyprogramming.com/2007/01/05/evolve-your-heirachy/). It was further popularized by Adam Martin in his extensive blogposts starting in 2007, with the title `Entity Systems are the future of MMOG development' (https://t-machine.org/index.php/2007/09/03/entity-systems-are-the-future-of-mmog-development-part-1/). He also established some of the modern terminoligy and concepts, like systems being first-class elements that actually hold (just) their code, components holding raw data and entities being just identifiers.

The main principle of ECS is composition over inheritance, which has many advantages not only related to performance but flexibility, making large projects easier to expand, debug and faster to change. Performance, if optimized right, can also be much higher for large/complex scenes, as systems iterate sequentially over the required components of relevant entities. As they are mostly saved close to another, iteration is also very cache-friendly. Additionally, ECS architectures can be parallelized very easily across multiple threads, by analyzing dependencies of systems and scheduling them to avoid multiple mutable access to the same components.

\section{Existing ECS libraries}

There are many ECS libraries already existing, many of them standalone for general purpose, but mostly games. There are so many in fact, that I cannot go into details comparing or grouping them, as that would be too much for the scope of this thesis. I will however go over some of the differences and data structures they use.

The first and one of the most important concepts is the memory layout of stored components. One approach is the \textit{archetype} concept, which my ECS uses as well. This means that any specific set of components an entity has at one time is called its \textit{archetype}, which usually is dynamic. For dynamic approaches, any component can be added or removed from an entity at runtime, which makes it change archetypes dynamically. Examples of ECS libraries with \textit{archetypes} include `flecs', `decs', `apecs', `hecs', `legion' and the Unity DOTS ecs package. Some of these ECS libraries combine that with a chunked approach, which splits the components of every archetype over multiple chunks, usually around 16 kB in size and memory alignment, to optimize cache efficiency while not having to reallocate the whole array/vector when its current size limit is reached. Another commonly used memory layout are \textit{sparse sets}, which in its core use two arrays/vectors; One is sparse and stores an index based on an item`s directly, which then points to an index in the dense array/vector where the actual data is. This dense data array/vector has no empty indices and is therefore ideal for iteration, while the sparse one is used to locate or add/remove single components. This concept is used, for example, in the ECS libraries `shipyard' and the well-known `EnTT', which is used in the game `Minecraft' (Mojang) among others. Other ECS libraries like `bevy_ecs' from the `Bevy' game engine use a hybrid approach, where in this case either a simple table layout can be used for fast iteration, where all components of a type are stored in a single column, or a sparse set.

\xxx{TODO: add github links}



\section{ECS in existing game engines}



\section{Categorizing existing game engines}


